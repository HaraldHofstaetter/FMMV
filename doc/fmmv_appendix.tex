\appendix
\chapter{FMMV User's Guide}
This user's guide describes the invocation of the \fmmv\ library
from user programs written in the programming languages C, Python,
or {\sc Matlab}. Instructions for building and
installing the library
can be found in the file \verb|INSTALL| in the distribution directory.

\fmmv\ provides routines  based on the Fast Multipole Method (FMM)
for the rapid evaluation of the
electrostatic (Coulomb)
potential and the corresponding force field due to 
three-dimensional distributions of charged particles and/or dipoles.
%
More precisely, there is a routine \verb|fmmvCoulomb3d| for the approximate evaluation of %the potential
\begin{equation}\label{pot}
	\phi(\mathbf{x}_i) = \sum_{j=1\atop j\neq i}^N
	\left(
	\frac{q_j}{\Vert\mathbf{x}_i-\mathbf{x}_j\Vert}
	+\frac{\mathbf{m}_j\cdot(\mathbf{x}_i-\mathbf{x}_j)}
	      {\Vert\mathbf{x}_i-\mathbf{x}_j\Vert^3}
	\right)    
	\quad\mbox{for $i=1,\dots,N$.}
\end{equation}
%at $\mathbf{x}_i$
Here, $q_j$ and $\mathbf{m}_j$ denote respectively 
the charge and dipole moment of the particle at $\mathbf{x}_j$.
Optionally, \verb|fmmvCoulomb3d| also computes an approximation of
the gradient
\begin{equation}\label{grad}
	\mbox{grad}\phi(\mathbf{x}_i),
\end{equation}
which (except for the sign) is the force field at $\mathbf{x}_i$ due to the potential $\phi$.

In (\ref{pot}) it is assumed that the locations of the charges are also
the locations where the potential is to be evaluated. In \fmmv\ the
possibilities of optimization due to this
additional symmetry %of this important special case for the applications 
are fully exploited, but alternatively it is possible to specify an  
independent array $\{\mathbf{y}_i\}$ of target locations where the potential shall be
evaluated. In this case \verb|fmmvCoulomb3d| computes an approximation to
\begin{equation}\label{pot_st}
	\phi(\mathbf{y}_i) = \sum_{j=1}^{N_{\mathrm{sources}}}
	\left(
	\frac{q_j}{\Vert\mathbf{y}_i-\mathbf{x}_j\Vert}
	+\frac{\mathbf{m}_j\cdot(\mathbf{y}_i-\mathbf{x}_j)}
	      {\Vert\mathbf{y}_i-\mathbf{x}_j\Vert^3}
	\right)      
	\quad\mbox{for $i=1,\dots,N_{\mathrm{targets}}$.} 
\end{equation}

\section{Calling FMMV from C}\label{cinterface}
\subsection{General Remarks}
The C interface of \fmmv\ consists of the routines
\begin{verbatim}
    fmmvCoulomb3d
    fmmvCoulomb3d_initialize
    fmmvCoulomb3d_evaluate
    fmmvCoulomb3d_finalize
    fmmvCoulomb3df
    fmmvCoulomb3df_initialize
    fmmvCoulomb3df_evaluate
    fmmvCoulomb3df_finalize
\end{verbatim}
where those routines with an additional '\verb|f|' are single precision
versions of the corresponding double precision routines without
the '\verb|f|'.
Programs calling one of these routines should include the header file
of \fmmv:
\begin{verbatim}
    #include<fmmv_coulomb3d.h>
\end{verbatim}
Such programs must be linked to the appropriate \fmmv\ library. On Unix-like
systems this is achieved by adding \verb|-lfmmv_coulomb3d -lm| 
(for double precision) or \verb|-lfmmv_coulomb3df -lm|
(for single precision) at the end of the link command.

In the sequel, we only describe the double precision routines.
Unless stated otherwise, the corresponding single precision routines 
behave exactly the same. Of course, for the single precision routines 
each occurrence of
the data type \verb|double| has to be replaced by \verb|float|.

\subsubsection*{Thread Safety}
All routines in \fmmv\ are thread-safe. This means it is safe to
call routines in parallel from multiple threads. This holds because
except for some precomputed arrays of coefficients which are modified only
at the first invocation of the library, no global data is shared across
internal subroutines of the library.
In particular, the routines  \verb|fmmvCoulomb3df_initialize|,
    \verb|fmmvCoulomb3df_evaluate|, and
    \verb|fmmvCoulomb3df_finalize|, which by necessity have
to share some data amongst them, do so by sharing only a handle, which has
to be passed as an argument, see Section~\ref{in_ev_fi}. However, sharing
the same handle across multiple threads is not  allowed.

\subsubsection*{Error Handling}
Generally, the routines of \fmmv\ are functions with a return value
of type \verb|char*| (pointer to char).
On success, they return a null pointer.
On failure, they return a pointer to a string, containing a
meaningful error message.
It is strongly recommended, always to check this return value and 
in case of failure to display the error message and abort the program.
Note that this is {\em not} done by default.


\subsection{{\tt fmmvCoulomb3d}}\label{allinone}
The \fmmv\ library is most straightforwardly invoked from C code by calling 
the routine \verb|fmmvCoulomb3d| (\verb|fmmvCoulomb3df| for single
precision). This is the recommended way if for a given distribution of particles,
the potential shall be evaluated only once
for a fixed set of charges or dipole moments.
For the case of several sets of charges, see Section~\ref{in_ev_fi}.
\subsubsection*{Function Declaration} % of \tt fmmvCoulomb3d}
\begin{verbatim}
    char* fmmvCoulomb3d(
        unsigned int NParticles, 
        double particles[][3], 
        double charges[],
        double dipoleMoments[][3],
        unsigned int NTargets, 
        double targets[][3],
        double pot[],      	
        double grad[][3], 	
        struct FmmvOptions *options, 
        struct FmmvStatistics *statistics);
\end{verbatim}

\subsubsection*{Arguments}
\begin{itemize}
\item {\verb|NParticles|} is the number of particles or the number
	of sources if independent targets are specified.
\item {\verb|particles|} is a pointer to an array containing the coordinates 
	of the particles. This means that the pointer \verb|particles|
	points to a memory block large enough to contain at least
	3*\verb|NParticles| elements of type \verb|double|. The arrangement
	of the elements is assumed to be the obvious one:

	\begin{tabular}{rl}\
		0: & $x$-coordinate of particle \#0\\
		1: & $y$-coordinate of particle \#0\\
		2: & $z$-coordinate of particle \#0\\
		3: & $x$-coordinate of particle \#1\\
		\vdots & \\
		\qquad
		3*\verb|NParticles|-1: & $z$-coordinate of particle \#\verb|NParticles|$-1$ 
	\end{tabular}	

	The array \verb|particles| can be allocated either on the stack
	using
\begin{verbatim}
    double particles[NParticles][3];
\end{verbatim}
	or on the heap using
\begin{verbatim}
    double (*particles)[3];
    ...
    particles = ((double(*))[3]) malloc(NParticles*3*sizeof(double));
\end{verbatim}
	The particular form of these declarations ensures that the 
	$j$-th Cartesian coordinate of the $i$-th particle
	($j\in\{0,1,2\}$, $i\in\{0,1,\dots$, \verb|NParticles|$-1\}$)
	can conveniently be accessed to by
	\verb|particles[|$i$\verb|][|$j$\verb|]|.\footnote{Note that in Fortran
	the order would be the opposite way round: 
	To get the same arrangement of elements, 
	the array would have to be defined by 
	
	\medskip
     \quad{\tt DOUBLE PRECISION PARTICLES(3, NPARTICLES)}
	\medskip

	\noindent The $j$-th coordinate  of the $i$-th particle is  
	accessed by {\tt PARTICLES(}$j${\tt ,}$i${\tt )}, where
	$j\in\{1,2,3\}$ and $i\in\{1,2,\dots$, {\tt NPARTICLES}$\}$.}

	Analogous considerations apply to the arrays
	\verb|dipoleMoments|, \verb|targets|, and \verb|grad|.
\item {\verb|charges|} is a pointer to an array such that 
	\verb|charges[|$i$\verb|]| is the charge of the $i$-th particle.
\item {\verb|dipoleMoments|} is a pointer to an array such that 
	\verb|dipoleMoments[|$i$\verb|][|$j$\verb|]| is the $j$-th Cartesian
	 component of the dipole moment of the $i$-th particle.

	At most one of the pointers \verb|charges| and 
	\verb|dipoleMoments| may be a null pointer:
	\begin{description}
	\item[] \verb|dipoleMoments==0|: all particles are monopole sources, i.e. $\mathbf{m}_j=0$ for all
	$j$ in (\ref{pot}) or (\ref{pot_st});
	\item[] \verb|charges==0|: all particles are dipole sources, i.e. $q_j=0$ for all
	$j$ in (\ref{pot}) or (\ref{pot_st}).
	\end{description}
\item {\verb|NTargets|} is the number of target locations. \verb|NTargets| is ignored if
	\verb|targets| is a null pointer.
\item {\verb|targets|} is a pointer to an array such that \verb|targets[|$i$\verb|][|$j$\verb|]| 
	is the $j$-th Cartesian coordinate of the $i$-th target location, i.e. the location
	where the potential (and optionally the gradient) shall be evaluated.

	If \verb|targets| is a null pointer, then \verb|particles| is used for the target locations,
	and an approximation to (\ref{pot}) instead of (\ref{pot_st}) is computed.
\item {\verb|pot|} is a pointer to an array of size at least \verb|NTargets| (or  \verb|NParticles| if
	\verb|targets| is a null pointer) elements of type \verb|double|.
	Note that \verb|pot| must not be a null pointer.
	After successful execution of \verb|fmmvCoulomb3d|, \verb|pot[|$i$\verb|]| contains
	the potential at the $i$-th target location.
\item {\verb|grad|} is a pointer to an array of size at least 3*\verb|NTargets| (or  3*\verb|NParticles| if
	\verb|targets| is a null pointer) elements of type \verb|double|.
	If \verb|grad| is a null pointer no gradients are computed. Otherwise, after successful execution of \verb|fmmvCoulomb3d|, \verb|grad[|$i$\verb|][|$j$\verb|]| contains the $j$-th Cartesian
	 component of the gradient at the $i$-th target location.
\item {\verb|options|} is a pointer to a \verb|FmmvOptions| structure for the fine-tuning
	of the algorithm, see Section~\ref{options} below.
	If \verb|options| is a null pointer default options are used.
\item {\verb|statistics|} is a pointer to a \verb|FmmvStatistics| structure, cf. 
	Section~\ref{statistics} below.
	If \verb|statistics| is a null pointer no statistics are provided. Otherwise, \verb|statistics|
	contains e.g. timing and memory usage information collected during the execution of the code.

\end{itemize}
\subsubsection*{Example of Usage}
\begin{verbatim}
    #include<stdio.h>    /* stderr, fprintf */
    #include<stdlib.h>   /* exit */
    #include<fmmv_coulomb3d.h>
    ...
    #define N_PARTICLES 10000
    ...
    {
        char *error_message;
        double particles[N_PARTICLES][3];
        double charges[N_PARTICLES];
        double pot[N_PARTICLES];
        ...
        /* some code for filling the arrays particles and charges */
        ...
        error_message = fmmvCoulomb3d(
            N_PARTICLES, particles, charges,
            0, 	   /* no dipoles */
            0, 0,  /* no independent target locations */
            pot,   
            0,     /* don't compute gradient */		
            0,     /* use default options */
            0);    /* don't generate statistics */
	
        if (error_message) {
            fprintf(stderr, "%s\n", error_message);
            exit(1);
        }	
        ...
    }
\end{verbatim}

\subsection{{\tt fmmvCoulomb3d\_initialize},
{\tt fmmvCoulomb3d\_evaluate}, % and
{\tt fmmvCoulomb3d\_finalize} }\label{in_ev_fi}
Equation (\ref{pot}) (with $\mathbf{m}_j=0$ for simplicity) can be written as 
\begin{equation}\label{yAx}
	y=Ax,
\end{equation}
with the vector $x\in\mathbb{R}^{N}$ of charges $q_j$, the (symmetric) matrix $A\in\mathbb{R}^{N\times N}$ given by
\begin{equation}
	A_{ij} = \frac{1}{\Vert\mathbf{x}_i-\mathbf{x}_j\Vert},
\end{equation}
and the result vector $y\in\mathbb{R}^{N}$ of the potentials $\phi(\mathbf{x}_i)$.
In many applications, however, the vector $y$ is given and the equation (\ref{yAx}) has to
be solved for the vector $x$.
This is usually done by some iterative method, in course of which many matrix-vector products of the form
(\ref{yAx}) have to be computed.

To handle such cases efficiently, \fmmv\
provides separate routines for the three main steps---initialization, 
evaluation, and finalization---of the algorithm:
\begin{itemize}
\item %The routine 
\verb|fmmvCoulomb3d_initialize| performs in advance
all initializations 
(like building up data structures, precomputing coefficients, etc.) 
depending only on the matrix $A$, i.e., on
the geometric distribution of the particles $\mathbf{x}_i$.
\item Subsequent calls to \verb|fmmvCoulomb3d_evaluate| evaluate (\ref{yAx}) for
several vectors $x$ %of  charges $q_i$,
without performing these initializations again.
\item Finally, \verb|fmmvCoulomb3d_finalize| frees all resources  that have been
allocated by
 \verb|fmmvCoulomb3d_initialize|.
\end{itemize}

\subsubsection*{Function Declarations}
\begin{verbatim}
    char* fmmvCoulomb3d_initialize(
        void** fmmvHandle,
        unsigned int NParticles, 
        double particles[][3], 
        unsigned int NTargets, 
        double targets[][3],
        int typeSources,  
        int computeGradient, 
        struct FmmvOptions *options, 
        struct FmmvStatistics *statistics); 

    char* fmmvCoulomb3d_evaluate(
        void* fmmvHandle,
        double charges[],
        double dipoleMoments[][3],
        double pot[],      	
        double grad[][3], 	
        struct FmmvStatistics *statistics); 

    char* fmmvCoulomb3d_finalize(
        void* fmmvHandle,
        struct FmmvStatistics *statistics); 
        \end{verbatim}
\subsubsection*{Arguments}
\begin{itemize}
	\item \verb|fmmvHandle| serves as a handle for the 
	internal communication between the routines.
	It is declared as a generic pointer (\verb|void*|) pointing
	to some internal data structure hidden to the user.	
	Note that \verb|fmmvHandle| is an output parameter
	for \verb|fmmvCoulomb3d_initialize| (causing the data type
	\verb|void**| in its declaration), whereas it is an
	input parameter for the other routines.

	\item \verb|NParticles|, \verb|particles|,
	      \verb|NTargets|, \verb|targets|: These arguments specifying
	      the distribution of the source and target locations have the 
	      same meaning as for \verb|fmmvCoulomb3d|, cf. Section~\ref{allinone}.		       Note that these data have to be supplied only for 
	      %the initialization routine 
	      \verb|fmmvCoulomb3d_initialize|.\footnote{The arrays
	      {\tt particles} and {\tt targets} are sorted
	      by {\tt fmmvCoulomb3d\_initialize}, and {\tt fmmvCoulomb3d\_evaluate}
	      can access to copies of these sorted arrays through {\tt fmmvHandle}.}
	      Again, \verb|target| may be a null pointer, in which case the points
	      in \verb|particles| are used for the target locations.
	      
	\item \verb|typeSources| denotes the kind of sources to be handled:
	\begin{itemize}
	\item[]\verb|typeSources==0|: only monopole charges;
	\item[]\verb|typeSources==1|: only dipoles;
	\item[]\verb|typeSources==2|: monopoles and dipoles.
	\end{itemize}
	Note that according to \verb|typeSource| the arguments
	\verb|charges| and \verb|dipoleMoments| of \verb|fmmvCoulomb3d_evaluate|
	may become mandatory, i.e., they must be pointers to 
	appropriate arrays and may not be null pointers.
	
	\item \verb|computeGradient| indicates whether gradients shall be
	computed. If  the quantity \verb|computeGradient| is non-zero, 
	the argument \verb|grad| of \verb|fmmvCoulomb3d_evaluate| is mandatory
	and must be a pointer to an appropriate output array.
	
	\item \verb|charges| and \verb|dipoleMoments| have the same meanings
	as in the case of \verb|fmmvCoulomb3d|, cf. Section~\ref{allinone}. 
	According to the value of the parameter \verb|typeSources| of
	\verb|fmmvCoulomb3d_initialize|,
	each of these arguments of \verb|fmmvCoulomb3d_evaluate| 
	 is either mandatory or ignored. 
	 
	\item \verb|pot| and \verb|grad|: Cf. Section~\ref{allinone}. If
	the parameter \verb|computeGradient| of \verb|fmmvCoulomb3d_initialize|
	was zero, \verb|grad| is not used and may be a null pointer.
	
	\item \verb|options| and \verb|statistics|: Cf. Section~\ref{allinone}.	
\end{itemize}

\subsubsection*{Example of Usage}
\begin{verbatim}
    #include<stdio.h>    /* stderr, fprintf */
    #include<stdlib.h>   /* exit */
    #include<fmmv_coulomb3d.h>
    ...
    #define N_PARTICLES 10000
    ...
    {
        void *fmmvHandle;
        char *error_message;
        double particles[N_PARTICLES][3];
        double charges[N_PARTICLES];
        double pot[N_PARTICLES];
        ...
        /* some code for filling the array particles */
        ...
        error_message = fmmvCoulomb3d_initialize(
            &fmmvHandle, N_PARTICLES, particles,
            0, 0,  /* no independent target locations */
            0,     /* only monopoles */
            0,     /* don't compute gradient */		
            0,     /* use default options */
            0);    /* don't generate statistics */

        if (error_message) {
            fprintf(stderr, "%s\n", error_message);
            exit(1);
        }	
        ...
        /* some code for filling the array charges */
        ...
        error_message = fmmvCoulomb3d_evaluate( 
            fmmvHandle, charges, 0, pot, 0, 0);	
	    
        if (error_message) {
            fprintf(stderr, "%s\n", error_message);
            exit(1);
        }	

        error_message = fmmvCoulomb3d_finalize(
            fmmvHandle, 0);
	
        if (error_message) {
            fprintf(stderr, "%s\n", error_message);
            exit(1);
        }	
        ...
    }
\end{verbatim}


\subsection{Fine-Tuning of \fmmv}\label{options}
For the fine-tuning of the algorithm 
a pointer to an options structure with the following
declaration may be supplied to the routines
\verb|fmmvCoulomb3d| and \verb|fmmvCoulomb_initialize|:
\begin{verbatim}
    struct FmmvOptions {
        int precision;           /* default 0 */
        double scale;            /* default 1.0 */
        int splitThreshold;       /* default depends on context */
        int splitTargetThreshold; /* default depends on context */
        int maxLevel;            /* default depends on context */
        int maxTargetLevel;      /* default depends on context */
        int directEvalThreshold;  /* default depends on context */
        int periodicBoundaryConditions;  /* default 0 */
        int useHilbertOrder;     /* default 0 */
        int useApproxInvSqrt;    /* default 0 */
        int useFarfieldNearfieldThreads; /* default 0 */
        int PAPIeventSet;        /* default PAPI_NONE */
    };
\end{verbatim}
A negative value for a field of this structure means that the 
default value indicated in the corresponding comment  
shall be taken.
For convenience, \fmmv\ provides the routine 
\begin{verbatim}
    struct FmmvOptions fmmvGetDefaultOptions(void);
\end{verbatim}
which returns an options structure with all fields set to
$-1$. This makes it easy to specify only 
a subset of all possible options. 
For example, in the following code segment
only the options \verb|precision| 
and \verb|periodicBoundaryConditions| are explicitly specified, all other options 
are set to their default values: 
\begin{verbatim}
    ...
    struct FmmvOptions options;
    ...
    options = fmmvGetDefaultOptions();
    options.precision = 1;
    options.periodicBoundaryConditions = 1;
    ...
\end{verbatim}    
The available options, sorted by categories, are listed in the following.

\subsubsection*{Options for Controlling Precision}
\begin{itemize}
   \item\verb|precision|
%\end{itemize}

The accuracy of %the version of 
the fast multipole method
implemented in \fmmv\ is determined by the parameters $p$
(length of multipole expansions) and $s$ (order of a certain internally
used quadrature formula, for details see Chapter~\ref{}).
The value of \verb|precision| selects
one of the three combinations of $p$ and $s$,
which are currently implemented in \fmmv,
according to the following table:
\begin{center}\begin{tabular}{c|rrc}
\verb|precision| &\quad $p$ & $s$ & $\varepsilon$ \\
\hline
0 &  6 & 8 & $10^{-3}$ \\ 
1 & 16 & 17 & $10^{-6}$ \\ 
2 & 23 & 26 & $10^{-9}$  
\end{tabular}\end{center}
Here $\varepsilon$ is approximately the expected accuracy. 
\verb|precision==2| is not implemented
%in \fmmv\ 
for single precision, because 
when calculations are carried out in
single precision, no accuracy higher than $\varepsilon=10^{-6}$ can be expected.

   \item\verb|useApproxInvSqrt|

The value of \verb|precision| affects only the accuracy of the
calculation of far field interactions. 
Interactions in the near field are directly calculated by (\ref{pot}) 
or (\ref{pot_st}) using an implementation
of the function $x\mapsto1/\sqrt{x}$, which calculates
its result to full floating-point precision.
Some processors however, like,  e.g., Pentium processors with SSE extensions or 
PowerPC processors with AltiVec extensions, provide approximations
to $x\mapsto1/\sqrt{x}$ in hardware, 
which---if available---can be utilized in \fmmv\ by setting
\verb|useApproxInvSqrt| to a value $\geq 1$. 
Such approximations  %to $x\mapsto1/\sqrt{x}$
are currently available only in single precision.
Their utilization leads to a significant decrease of the run time,
but their accuracy is compatible with \verb|precision==0| only, i.e.,
$\varepsilon=10^{-3}$.
\end{itemize}
 

\subsubsection*{Options for Controlling the Oct-Tree Data Structure}
\begin{itemize}
   \item\verb|splitThreshold|
   \item\verb|maxLevel|
   \item\verb|splitTargetThreshold|
   \item\verb|maxTargetLevel |
   \smallskip

\verb|splitThreshold==0| requires the oct-tree data structure to be
built according to the 
{\em non-adaptive FMM.} Boxes are recursively divided into eight child boxes 
until \verb|maxLevel| levels of boxes have been obtained.
Leaf boxes occur only at the highest (finest) level.
For the non-adaptive FMM the values of \verb|splitTargetThreshold|
and \verb|maxTargetLevel| are ignored, even if independent target
locations are specified.

\verb|splitThreshold>0| requires the oct-tree data structure to be
built according to the 
{\em adaptive FMM.} Boxes that are not already in level 
\verb|maxLevel| 
are recursively divided into eight child boxes 
as long as they contain more than \verb|splitThreshold| particles.
Leaf boxes may occur at all levels.
If independent target locations are specified, the options 
\verb|splitThreshold| and \verb|maxLevel| only apply to the
assignment of sources to source boxes.
Assignment of targets to target boxes is controlled 
analogously by the
options \verb|splitTargetThreshold| and \verb|maxTargetLevel|.

It should be mentioned that for \verb|splitThreshold==0| and
\verb|maxLevel==0| the FMM is not involved at all, 
all interactions are computed directly.
This may be important for comparison purposes, because for
this direct algorithm the same highly optimized routines of \fmmv\ are
used. Usually, not much effort is spent on the optimization
of the direct algorithm, which makes the comparison with the FMM 
somewhat unfair.

\item\verb|useHilbertOrder|

While building the oct-tree data structure, 
all particles are sorted
in such a way that for each box the particles belonging 
to this box are contiguously stored in memory.
By default, this goal is achieved by sorting the particles according
to  standard {\em Molton order}, cf. Chapter~\ref{}.

Alternatively, setting \verb|useHilbertOrder|  to a value
$\geq 1$ requires  the particles to be sorted along a {\em Hilbert space filling
curve}, which often leads to better performance due
to improved cache utilization. 

   \item\verb|scale|

   Before building the oct-tree data structure, all particles are scaled
   to fit into  $[0,1]^3$, i.e., the unit cube.\footnote{Actually the 
   particles are scaled to fit into the cube $[1/2,1]^3$, which
   offers some technical advantages, see Chapter~\ref{}.}
   An additional scaling factor $\in (1/2,1]$
   may be specified via \verb|scale|.
   For the non-adaptive FMM this has the effect, that for a given
   number of levels, less leaf boxes are generated 
   and accordingly more particles belong to each leaf box.
   Bearing in mind that for uniform particle distributions
   the average number of particles per leaf box
   is an essential parameter to be optimized, this 
   additional scaling allows 
   much finer control of this parameter than by controlling it 
   only by means of the number of levels. 
   More information on this idea can be found in \cite{fractional}.
   
   
\end{itemize}

\subsubsection*{Far Field/Near Field Parallelism}
\begin{itemize}
    \item\verb|useFarfieldNearfieldThreads|

    By the very nature of the FMM, the two steps
    \begin{itemize}
    \item calculating all far field interactions and
    \item calculating all near field interactions 
    \end{itemize}
    are essentially mutually independent. Thus, the
    order in which they are performed does not matter,
    they can even be performed in parallel.
    As experience shows, balancing computing time among these
    two steps often minimizes the total computing time. Consequently,
    doing these steps in parallel seems even more desirable.
    In \fmmv\ this parallelism is implemented using POSIX threads,
    and is enabled by setting \verb|useFarfieldNearfieldThreads|
    to a value $\geq 1$.
    Of course, this only makes sense if more than one processor 
    (or a dual core processor) is available.
    
    
\end{itemize}

\subsubsection*{Periodic Boundary Conditions}
\begin{itemize}
    \item\verb|periodicBoundaryConditions|

\fmmv\ supports the use of periodic boundary conditions. 
If \verb|periodicBoundaryConditions| is set to a value
$\geq 1$, the appropriate routines of \fmmv\ compute
an approximation to
\begin{equation}\label{pot_periodic}
	\phi(\mathbf{x}_i) = 
	\sum_{\mathbf{n}\in\mathbb{Z}^3}
	\sum_{j=1\atop \mbox{\tiny$j\neq i$ for $\mathbf{n}=0$}}^N
	\left(
	\frac{q_j}{\Vert\mathbf{x}_i-\mathbf{x}_j-\mathbf{n}\Vert}
	+\frac{\mathbf{m}_j\cdot(\mathbf{x}_i-\mathbf{x}_j-\mathbf{n})}
	      {\Vert\mathbf{x}_i-\mathbf{x}_j-\mathbf{n}\Vert^3}
	\right)
\end{equation}	
for $i=1,\dots,N$, instead of an approximation to (\ref{pot}).
This means that all given charges in the basic box (unit cube) are replicated
periodically to all boxes obtained by translation of the basic box due
to all possible vectors $\mathbf{n}\in\mathbb{Z}^3$.
Note that the outer sum in (\ref{pot_periodic})  is only
conditionally convergent. Consequently,
some convention about the order of summation is
necessary. This is explained in more detail in Section~\ref{}, see also
\cite{periodic}.

\end{itemize}

\subsubsection*{Counting Events Using PAPI}
\begin{itemize}
    \item\verb|PAPIeventSet|

    The PAPI library\footnote{PAPI is available at
    {\tt http://icl.cs.utk.edu/papi/}.}
    provides a standard interface 
    for accessing hardware performance counters available 
    on most modern microprocessors.
    If \fmmv\ has been compiled with PAPI support enabled
    ( , 
    A
    \begin{verbatim}
    struct FmmvOptions options = fmmvGetDefaultOptions();
    ...
    PAPI_library_init(PAPI_VER_CURRENT);
    options.PAPIeventSet = PAPI_NULL;
    PAPI_create_eventset(&options.PAPIeventSet);
    ...
    PAPI_add_event(options.PAPIeventSet, PAPI_TOT_INS);
    ...
    \end{verbatim}
The functions of the PAPI library have  return values indicating the
success of the respective operations. These have been ignored
in the above code segment for simplicity. This, of course, is not
recommended for actual usage of the PAPI library.
\end{itemize}



\subsection{Statistics}\label{statistics}
If a pointer to a statistics structure whose
declaration is given below, is supplied to the
routines of \fmmv\, timing and memory usage information
is collected during the execution of the code.
This information can conveniently be printed
in readable form
to standard output by  calling the routine
\begin{verbatim}  
    void printFmmvStatistics(struct FmmvStatistics *statistics);
\end{verbatim}
The declaration of the statistics 
structure \verb|FmmvStatistics| is as follows:
\begin{verbatim}
    #define STAT_MAX 20
    #define MAX_NUM_PAPI_EVENTS 10

    struct FmmvStatistics {
        double time[STAT_MAX];
        double etime[STAT_MAX];

        int PAPIeventSet;
        long long int PAPIvalues[STAT_MAX][MAX_NUM_PAPI_EVENTS];

        unsigned int maxAllocatedMemory;

        int p;
        int s;

        int noOfParticles;
        int noOfTargets;
        int noOfSourceLevels;
        int noOfTargetLevels;

        int noOfSourceBoxes;
        int noOfTargetBoxes;
        int noOfSourceLeafBoxes;
        int noOfTargetLeafBoxes;
        float averageNoOfParticlesPerLeafBox;
        float averageNoOfTargetsPerLeafBox;
    
        int noOfParticlesInLevel[52];
        int noOfTargetsInLevel[52];
        int noOfSourceBoxesInLevel[52];
        int noOfTargetBoxesInLevel[52];
        int noOfSourceLeafBoxesInLevel[52];
        int noOfTargetLeafBoxesInLevel[52];
        float averageNoOfParticlesPerLeafBoxInLevel[52];
        float averageNoOfTargetsPerLeafBoxInLevel[52];
    };
\end{verbatim}
The meaning of most of the fields should be self-explanatory, so only
some of them are described in the following.
\begin{itemize}
  \item \verb|p| and \verb|s| are the  values used for the parameters 
  mentioned in Section~\ref{options}, which determine the accuracy of
  the FMM.
  \item \verb|maxAllocatedMemory| is the maximum size of memory in bytes,
  allocated during the execution of the code. This value
  is determined by keeping track to each call of the system routines
  \verb|malloc| and \verb|free| (or to similar routines supporting
  aligned memory allocation).
  This also allows an internal check, whether the entire 
  allocated memory has been properly deallocated.

  \item \verb|time| and \verb|etime|:
  These arrays provide timing information
  itemized to substeps of the FMM algorithm.
  The timing in \verb|time| (processor time) is
  obtained via the system routine \verb|clock|,
  whereas the value provided in \verb|etime| (elapsed time)
  is obtained via the system routine \verb|gettimeofday|.
  The available steps are given by the following 
  enum constants, such that, e.g., \verb|time[STAT_M2M]| gives
  the timing of the M2M substep of the FMM:
\begin{verbatim}
    enum StatStep {
        STAT_TOTAL,
        STAT_BUILD_TREE, 
        STAT_GEN_M,
        STAT_M2M, 
        STAT_M2L, 
        STAT_L2L,
        STAT_EVAL_L,
        STAT_LIST1, 
        STAT_LIST3, 
        STAT_LIST4, 
        STAT_LIST34,
        STAT_FARFIELD, 
        STAT_NEARFIELD,
        STAT_INITIALIZE,
        STAT_EVALUATE,
        STAT_FINALIZE,
        _STAT_LAST_ /* mark last entry, do not remove! */	
    };
\end{verbatim}
  If some substep is not applicable, the corresponding values are set
  to $-1$.
  \item \verb|PAPIvalues|:
\end{itemize}  

\section{Calling FMMV from Python}\label{pythoninterface}
\subsection{General Remarks}
\fmmv\ provides an interface for the programming language
Python with similar functionality as the C interface, 
cf. Section~\ref{cinterface}.
The interface depends on the package \verb|numarray|\footnote{{\tt numarray}
is available at {\tt http://www.stsci.edu/resources/software\_hardware/numarray}.},
which
adds efficient array manipulation capabilities to the Python programming
language.
Python programs calling one of the double precision
routines of \fmmv\ must import
the modules \verb|numarray| and \verb|fmmv|:
\begin{verbatim}
    from numarray import *
    from fmmv import *
\end{verbatim}
For single precision, the module \verb|fmmvf| has to be imported instead
of \verb|fmmv|:
\begin{verbatim}
    from numarray import *
    from fmmvf import *
\end{verbatim}
In either case, this gives access to the routines
\begin{verbatim}
    coulomb3d
    coulomb3d_initilize
    coulomb3d_evaluate
    coulomb3d_finalize
\end{verbatim}
which calculate in double or single precision, depending on whether they 
were imported via \verb|fmmv| or \verb|fmmvf|.
\subsubsection*{Error Handling}
In case of error the routines of the modules \verb|fmmv|
and \verb|fmmvf|
throw an exception, 
which---if not caught by an exception handler---causes 
the calling program to abort after printing an error message.

\subsection{{\tt coulomb3d}}\label{python_coulomb3d}
The most basic usage of \verb|coulomb3d| looks like
\begin{verbatim}
    pot = coulomb3d(particles, charges)
\end{verbatim}
Here,  {\tt particles} and {\tt charges} are {\tt numarray} arrays containing 
the coordinates of the particles and the corresponding charges, respectively.
They are generated by, e.g.,\footnote{Notice the order of the dimensions
{\tt(NParticles, 3)} in the definition of {\tt particles} (and similarly for
{\tt dipoleMoments}).
This resembles the fact that in
Python %together with the package \verb|numarray| 
arrays are stored {\em row-wise}.
The $j$-th coordinate of the $i$-th particle is accessed by
{\tt particles[}$i${\tt ,}$j${\tt ]}, 
where $j\in\{0,1,2\}$ and $i\in\{0,\dots, $ {\tt nparticles}$-1\}$.
}
\begin{verbatim}
    particles = zeros((NParticles, 3), type=Float64)
    charges = zeros(NParticles, type=Float64)
\end{verbatim}
where \verb|NParticles| denotes the number of particles, and \verb|type|
indicates the type of arrays to be generated (\verb|Float32| for single
and \verb|Float64| for double precision), and have then to be filled with
their intended values by some Python code.  

After successful execution of \verb|coulomb3d|, \verb|pot| is 
%a \verb|numarray| 
an array containing an approximation to the potential (\ref{pot}) without
dipole sources, i.e., $\mathbf{m}_i=0$.

\subsubsection*{Dipole Sources}
The usage of \verb|coulomb3d| for 
dipole sources, whose dipole moments are supplied by the array
\verb|dipoleMoments|, 
looks like
\begin{verbatim}
    pot = coulomb3d(particles, dipoleMoments=dipoleMoments)
\end{verbatim}
or
\begin{verbatim}
    pot = coulomb3d(particles, charges, dipoleMoments=dipoleMoments)
\end{verbatim}
where in the latter case the particles also carry monopole charges.
\subsubsection*{Computing Gradients}
The additional argument \verb|computeGradient=True| requires
\verb|coulomb3d| to compute  approximations to the gradients
(\ref{grad}):
\begin{verbatim}
    (pot, grad) = coulomb3d(..., computeGradient=True)
\end{verbatim}
Here and in the following the dots indicate further required 
arguments, e.g., the ones described above.
\subsubsection*{Independent Target Locations}
If the potential and optionally  the gradient shall be computed
at independent target locations, the Cartesian coordinates of these locations 
have to be supplied by an appropriate array \verb|targets|:
\begin{verbatim}
    pot = coulomb3d(..., targets=targets)
\end{verbatim}
or
\begin{verbatim}
    (pot, grad) = coulomb3d(..., targets=targets, computeGradient=True)
\end{verbatim}
\subsubsection*{Further Options}
    For the fine-tuning of the algorithm \verb|coulomb3d| supports
    the following options, whose meanings are exactly the same as  for
    the corresponding options of the C interface of \fmmv\ described in Section~\ref{options}:
\begin{verbatim}
    precision
    useApproxInvSqrt
    splitThreshold
    maxLevel
    splitTargetThreshold
    maxTargetLevel
    useHilbertOrder
    scale
    useFarfieldNearfieldThreads
    periodicBoundaryConditions
\end{verbatim}
Options are set according to the scheme
\begin{verbatim}
    fmmvHandle = coulomb3d(..., option1=value1, option2=value2,...)
\end{verbatim}
For missing options, or for options explicitly set to the value \verb|None|, default values 
as described in Section~\ref{options} 
are taken.
\subsubsection*{Statistics}
The additional argument \verb|getStatistics=True| requires
\verb|coulomb3d| to collect timing and memory usage information:
\begin{verbatim}
    (pot, stat) = coulomb3d(..., getStatistics=True)
\end{verbatim}
or
\begin{verbatim}
    (pot, grad, stat) = coulomb3d(..., computeGradient=True, getStatistics=True)
\end{verbatim}
Here, the output parameter \verb|stat| is a dictionary with the following strings as keys:
\begin{verbatim}
    "p"
    "s"
    "maxAllocatedMemory"
    "noOfParticles"
    "noOfTargets"
    "noOfSourceLevels"
    "noOfTargetLevels"
    "noOfSourceBoxes"
    "noOfTargetBoxes"
    "noOfSourceLeafBoxes"
    "noOfTargetLeafBoxes"
    "averageNoOfParticlesPerLeafBox"
    "averageNoOfTargetsPerLeafBox"
    "noOfParticlesInLevel"
    "noOfTargetsInLevel"
    "noOfSourceBoxesInLevel"
    "noOfTargetBoxesInLevel"
    "noOfSourceLeafBoxesInLevel"
    "noOfTargetLeafBoxesInLevel"
    "averageNoOfParticlesPerLeafBoxInLevel"
    "averageNoOfTargetsPerLeafBoxInLevel"
    "totalTime"
    "buildTreeTime"
    "genMTime"
    "M2MTime"
    "M2LTime"
    "L2LTime"
    "evalLTime"
    "list1Time"
    "list3Time"
    "list4Time"
    "list34Time"
    "farfieldTime"
    "nearfieldTime"
    "initializeTime"
    "evaluateTime"
    "finalizeTime"
\end{verbatim}
The meanings of the keys are self-explanatory or explained
in Section~\ref{statistics}. Keys ending with \verb|"Time"|
provide the running times of the corresponding substeps of the 
FMM obtained via the system routine \verb|gettimeofday|, or
are missing in the dictionary \verb|stat| if the corresponding
substeps are not applicable.
For example, \verb|stat["M2MTime"]| gives the running time
of the M2M substep of the FMM, and
\verb|stat["noOfSourceBoxesInLevel"][3]| gives the
number of source boxes in level~3 of the oct-tree data structure.
    
\subsection{{\tt coulomb3d\_initialize}, 
{\tt coulomb3d\_evaluate},\\
{\tt coulomb3d\_finalize}}
Analogously to the C interface described in Section~\ref{in_ev_fi},
the Python interface of  \fmmv\ provides separate routines
\begin{verbatim}
    coulomb3d_initialize
    coulomb3d_evaluate
    coulomb3d_finalize
\end{verbatim}
for the three main steps---initialization, evaluation, and finalization---of the algorithm.  
Now, the basic usage looks like
\begin{verbatim}
    fmmvHandle = coulomb3d_initialize(particles)
    ...
    pot = coulomb3d_evaluate(fmmvHandle, charges)
    ...
    coulomb3d_finalize(fmmvHandle)
\end{verbatim}
The usage is similar to the usage of \verb|coulomb3d| (cf. Section~\ref{python_coulomb3d}), but
with the arguments 
appropriately distributed among the routines \verb|coulomb3d_initialize| and
\verb|coulomb3d_evaluate|, and with a handle \verb|fmmvHandle| serving for
the internal communication between the routines.
\subsubsection*{Dipole Sources}
The kind of sources to be handled can be determined by the argument 
\verb|typeSources| of \verb|coulomb3d_initialize|:
\begin{itemize}
	\item \verb|typeSources=0| for monopole charges only. As shown above, 
	this does not have to be explicitly specified.
	\item \verb|typeSources=1| for dipole charges only:
\begin{verbatim}
    fmmvHandle = coulomb3d_initialize(particles, typeSources=1)
    ...
    pot = coulomb3d_evaluate(fmmvHandle, dipoleMoments=dipoleMoments)
    ...
\end{verbatim}
	\item \verb|typeSources=2| for monopole and dipole charges:
\begin{verbatim}
    fmmvHandle = coulomb3d_initialize(particles, typeSources=2)
    ...
    pot = coulomb3d_evaluate(fmmvHandle, charges, dipoleMoments=dipoleMoments)
    ...
\end{verbatim}
\end{itemize}
In the latter two cases the argument \verb|dipoleMoments| of \verb|coulomb3d_evaluate|
is of course mandatory.
\subsubsection*{Computing Gradients}
The argument \verb|computeGradient=True| of \verb|coulomb3d_initialize| requires
\verb|coulomb3d_evaluate| to compute approximations to the gradients (\ref{grad}):
\begin{verbatim}
    fmmvHandle = coulomb3d_initialize(..., computeGradient=True)
    ...
    (pot, grad) = coulomb3d_evaluate(fmmvHandle, ...)
    ...
\end{verbatim}
\subsubsection*{Independent Target Locations}
Independent target locations supplied to \verb|coulomb3d_initialize|,
\begin{verbatim}
    fmmvHandle = coulomb3d_initialize(..., targets=targets)
\end{verbatim}
cause subsequent calls of \verb|coulomb3d_evaluate| to calculate their results
at these given locations.
\subsubsection*{Further Options}
Options can be set at time of initialization only:
\begin{verbatim}
    fmmvHandle = coulomb3d_initialize(..., option1=value1, option2=value2,...)
\end{verbatim}
These options then also apply to subsequent calls of \verb|coulomb3d_evaluate|.
\subsubsection*{Statistics}
If \verb|getStatistics=True| is specified for \verb|coulomb3d_initialize|,
then timing and memory usage information is collected during execution of
{\em all} of the three steps initialization, evaluation, and finalization:
\begin{verbatim}
    (fmmvHandle, stat) = coulomb3d(..., getStatistics=True)
    ...
    (pot, stat) = coulomb3d_evaluate(fmmvHandle, ...) 
    # or possibly (pot, grad, stat) = ...
    ...
    stat = coulomb3d_finalize(fmmvHandle, ...)
\end{verbatim}

\section{Calling FMMV from MATLAB}
\subsection{General Remarks}
\fmmv\ provides an interface for easy access of the library from within {\sc Matlab}.
This interface has been modeled after the Python interface described 
in Section~\ref{pythoninterface} as far as it is
possible and compatible with the
conventions used in {\sc Matlab}.
Consequently, also the documentation of the {\sc Matlab} interface given in this
section follows closely Section~\ref{pythoninterface}.

Currently the {\sc Matlab} interface of \fmmv\ provides only routines using
double precision, which are the following:
\begin{verbatim}
    fmmvcoulomb3d
    fmmvcoulomb3d_initialize
    fmmvcoulomb3d_evaluate
    fmmvcoulomb3d_finalize
\end{verbatim}
These routines are readily available as soon as their corresponding program
files (shared libraries) are located in some directory on the {\sc Matlab} path.
No explicit initialization of the library 
like importing some modules is necessary.
\subsubsection*{Error Handling}
In case of error the routines of the {\sc Matlab} interface of \fmmv\
throw an exception, 
which---if not caught by an exception handler---
causes the calling program to stop execution after printing an error message.


\subsection{\tt fmmvcoulomb3d}
The basic usage of \verb|fmmvcoulomb3d| is
\begin{verbatim}
    pot = fmmvcoulomb3d(particles, charges)
\end{verbatim}
where the arrays \verb|particles| and \verb|charges| 
contain the coordinates of
the particles and the corresponding charges, respectively.
These arrays are generated by, e.g.,
\begin{verbatim}
    particles = zeros(3, nparticles)
    charges = zeros(1, nparticles)
\end{verbatim}
where \verb|nparticles| denotes the number of particles.\footnote{Notice
the order of
the dimensions of the array {\tt particles}, which resembles the
fact that in {\sc Matlab} arrays are stored {\em column-wise}.
The $j$-th coordinate of the $i$-th particle is accessed by
{\tt particles(}$j${\tt ,}$i${\tt )},
where $j\in\{1,2,3\}$ and $i\in\{1,\dots, $ {\tt nparticles}$\}$.
}

\subsubsection*{Dipole Sources}
For dipole sources with dipole moments
given by the array \verb|dipoleMoments|
the usage of \verb|fmmvcoulomb3d| is
\begin{verbatim}
    pot = fmmvcoulomb3d(particles, [], dipoleMoments)
\end{verbatim}
or
\begin{verbatim}
    pot = fmmvcoulomb3d(particles, charges, dipoleMoments)
\end{verbatim}
where in the first case the empty matrix \verb|[]| serves as a placeholder
for the absent monopole charges,
and in the latter case the particles also carry monopole charges.
\subsubsection*{Computing Gradients}
Gradients are computed, if a second output parameter \verb|grad| 
is supplied to the call of the routine \verb|fmmvcoulomb3d|:
\begin{verbatim}
    [pot, grad] = fmmvcoulomb3d(...)
\end{verbatim}
\subsubsection*{Independent Target Locations}
Independent target locations can be supplied by a fourth argument
\verb|targets|:
\begin{verbatim}
    pot = fmmvcoulomb3d(particles, charges, dipolemoments, targets)
\end{verbatim}
or
\begin{verbatim}
    [pot, grad] = fmmvcoulomb3d(particles, charges, dipolemoments, targets)
\end{verbatim}
Here the potentials and gradients are computed at the locations
given by the array \verb|targets|, not at those locations given by 
the array \verb|particles|.
\subsubsection*{Further Options}
For the fine-tuning of the algorithm, an options structure can be supplied
to \verb|fmmvcoulomb3d| as a fifth argument:
\begin{verbatim}
    [pot, grad] = fmmvcoulomb3d(particles, charges, dipolemoments, targets, options)
\end{verbatim}
For the creation of such options structures \fmmv\ provides
the {\sc Matlab} function \verb|fmmvset| with the following syntax:\footnote{This
method of handling options is modeled after the one used by the ODE solvers of
{\sc Matlab}.}
\begin{verbatim}
    options = fmmvset('option1', value1, 'option2', value2, ...)
\end{verbatim}
Here \verb|option1|, \verb|option2|,\dots must be one of the following options,
whose meanings are exactly the same as for the corresponding options of
the C interface of \fmmv\ described in Section~\ref{options}:
\begin{verbatim}
    precision
    useApproxInvSqrt
    splitThreshold
    maxLevel
    splitTargetThreshold
    maxTargetLevel
    useHilbertOrder
    scale
    useFarfieldNearfieldThreads
    periodicBoundaryConditions
    printStatistics
\end{verbatim}
Any unspecified options have default values as described in Section~\ref{options}.
\subsubsection*{Statistics}
One of the options described above is \verb|printStatistics|. If this is set
to a non-zero value, timing and memory usage information collected during
the execution of the code is printed in readable form to standard output.
\subsection{{\tt fmmvcoulomb3d\_initialize}, 
{\tt fmmvcoulomb3d\_evaluate},\\
{\tt fmmvcoulomb3d\_finalize}}
Analogously to the C interface described in Section~\ref{in_ev_fi},
the {\sc Matlab} interface of  \fmmv\ provides separate routines
\begin{verbatim}
    fmmvcoulomb3d_initialize
    fmmvcoulomb3d_evaluate
    fmmvcoulomb3d_finalize
\end{verbatim}
for the three main steps---initialization, evaluation, and finalization---of 
the algorithm.  
The basic usage now looks like
\begin{verbatim}
    fmmvhandle = fmmvcoulomb3d_initialize(particles)
    ...
    pot = fmmvcoulomb3d_evaluate(fmmvhandle, charges)
    ...
    fmmvcoulomb3d_finalize(fmmvhandle)
\end{verbatim}
Here, \verb|fmmvHandle| serves for the internal communication 
between the routines.
\subsubsection*{Dipole Sources}
The kind of sources to be handled can be determined by a second argument 
\verb|type| to \verb|fmmvcoulomb3d_initialize|:
\begin{itemize}
	\item \verb|type=0| for monopole charges only. As shown above, 
	this is the default and does not have to be explicitly specified.
	\item \verb|type=1| for dipole charges only:
\begin{verbatim}
    fmmvhandle = fmmvcoulomb3d_initialize(particles, 1)
    ...
    pot = fmmvcoulomb3d_evaluate(fmmvhandle, [], dipolemoments)
    ...
\end{verbatim}
	In this case the second argument of \verb|fmmvcoulomb3d_evaluate|
	is ignored and can thus be, e.g.,  
	the empty matrix \verb|[]|.
	\item \verb|type=2| for monopole and dipole charges:
\begin{verbatim}
    fmmvhandle = fmmvcoulomb3d_initialize(particles, 2)
    ...
    pot = fmmvcoulomb3d_evaluate(fmmvhandle, charges, dipolemoments)
    ...
\end{verbatim}
\end{itemize}
In the latter two cases the argument \verb|dipolemoments| of \verb|fmmvcoulomb3d_evaluate|
is of course mandatory.

\subsubsection*{Computing Gradients}
A third non-zero argument of \verb|fmmvcoulomb3d_initialize| requires
\verb|fmmvcoulomb3d_evaluate| to compute gradients. 
\begin{verbatim}
    fmmvhandle = fmmvcoulomb3d_initialize(charges, type, 1)
    ...
    [pot, grad] = fmmvcoulomb3d_evaluate(fmmvhandle, charges, dipolemoments)
    ...
\end{verbatim}
In this case, the second output parameter \verb|grad| of 
\verb|fmmvcoulomb3d_evaluate|
is mandatory.
\subsubsection*{Independent Target Locations}
Independent target locations supplied to \verb|fmmvcoulomb3d_initialize|,
\begin{verbatim}
    fmmvhandle = fmmvcoulomb3d_initialize(particles, type, grad, targets)
\end{verbatim}
cause subsequent calls of \verb|fmmvcoulomb3d_evaluate| to calculate their results
at these given locations.
\subsubsection*{Further Options}
An options structure can be supplied at time of initialization only:
\begin{verbatim}
    fmmvhandle = fmmvcoulomb3d_initialize(particles, type, grad, targets, options)
\end{verbatim}
The corresponding options then also apply to subsequent calls of 
\verb|fmmvcoulomb3d_evaluate|.

